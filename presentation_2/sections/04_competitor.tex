% -------------------- %

\begin{frame}{\lpn}
\framesubtitle{Competitor}

\begin{itemize}
    \item \lp{} for Non-stationary and Imbalanced Environments
    \item modified algorithm of \lpc
    \begin{itemize}
        \item employs a different penalty constraint that forces the algorithm to balance predictive accuracy on all classes
        \item uses a bagging based sub-ensemble for the minority class oversampling
    \end{itemize}
\end{itemize}

\begin{block}{For more details}
\small{Ditzler, Gregory \& Polikar, Robi. (2013). Incremental Learning of Concept Drift from Streaming Imbalanced Data. Knowledge and Data Engineering, IEEE Transactions on. 25. 2283-2301. 10.1109/TKDE.2012.136.} 
\end{block}

\end{frame}

% -------------------- %

\begin{frame}{DWMIL vs. \lpn}
\framesubtitle{Forest Covertype}

\begin{itemize}
    \item DWMIL performs better at this real-world data set
\end{itemize}

\begin{table}[h]
    \centering
    \begin{tabular}{ | l | l | l | }
    \hline
    metric & DWMIL & \lpn \\ \hline \hline
    gm & 0.9222 & 0.8984 \\ \hline
    f1 & 0.8040 & 0.7838 \\ \hline
    auc & 0.9211 & 0.8928 \\ \hline
    rec & 0.9129 & 0.8616 \\ \hline
    \end{tabular}
\end{table}

\end{frame}

% -------------------- %

\begin{frame}{DWMIL vs. \lpn}
\framesubtitle{Moving Gaussian}

\begin{itemize}
    \item both methods perform equally good
\end{itemize}

\begin{table}[h]
    \centering
    \begin{tabular}{ | l | l | l | }
    \hline
    metric & DWMIL & \lpn \\ \hline \hline
    gm & 0.9115 & 0.9282 \\ \hline
    f1 & 0.9116 & 0.9291 \\ \hline
    auc & 0.9115 & 0.9281 \\ \hline
    rec & 0.9114 & 0.9252 \\ \hline
    \end{tabular}
\end{table}

\end{frame}

% -------------------- %

\begin{frame}{DWMIL vs. \lpn}
\framesubtitle{SEA}

\begin{itemize}
    \item \lpn{} performs better at this data set
\end{itemize}

\begin{table}[h]
    \centering
    \begin{tabular}{ | l | l | l | }
    \hline
    metric & DWMIL & \lpn \\ \hline \hline
    gm & 0.7705 & 0.9231 \\ \hline
    f1 & 0.8155 & 0.9350 \\ \hline
    auc & 0.7705 & 0.9217 \\ \hline
    rec & 0.7848 & 0.9503 \\ \hline
    \end{tabular}
\end{table}

\end{frame}

% -------------------- %

\begin{frame}{DWMIL vs. \lpn}
\framesubtitle{Hyper Plane}

\begin{itemize}
    \item \lpn{} performs better at this data set
\end{itemize}

\begin{table}[h]
    \centering
    \begin{tabular}{ | l | l | l | }
    \hline
    metric & DWMIL & \lpn \\ \hline \hline
    gm & 0.6460 & 0.8464 \\ \hline
    f1 & 0.6612 & 0.8558 \\ \hline
    auc & 0.6466 & 0.8474 \\ \hline
    rec & 0.6224 & 0.8263 \\ \hline
    \end{tabular}
\end{table}

\end{frame}

% -------------------- %

\begin{frame}{DWMIL vs. \lpn}
\framesubtitle{Checkerboard}

\begin{itemize}
    \item \lpn{} performs better at this data set
\end{itemize}

\begin{table}[h]
    \centering
    \begin{tabular}{ | l | l | l | }
    \hline
    metric & DWMIL & \lpn \\ \hline \hline
    gm & 0.6508 & 0.8631 \\ \hline
    f1 & 0.6571 & 0.8633 \\ \hline
    auc & 0.6510 & 0.8626 \\ \hline
    rec & 0.6374 & 0.8793 \\ \hline
    \end{tabular}
\end{table}

\end{frame}

% -------------------- %

\begin{frame}{DWMIL vs. \lpn}
\framesubtitle{Electricity}

\begin{itemize}
    \item both methods perform equally good
\end{itemize}

\begin{table}[h]
    \centering
    \begin{tabular}{ | l | l | l | }
    \hline
    metric & DWMIL & \lpn \\ \hline \hline
    gm & 0.7550 & 0.7415 \\ \hline
    f1 & 0.7197 & 0.6988 \\ \hline
    auc & 0.7557 & 0.7489 \\ \hline
    rec & 0.7241 & 0.6474 \\ \hline
    \end{tabular}
\end{table}

\end{frame}

% -------------------- %

\begin{frame}{DWMIL vs. \lpn}
\framesubtitle{Weather}

\begin{itemize}
    \item both methods perform equally good
\end{itemize}

\begin{table}[h]
    \centering
    \begin{tabular}{ | l | l | l | }
    \hline
    metric & DWMIL & \lpn \\ \hline \hline
    gm & 0.7405 & 0.7247 \\ \hline
    f1 & 0.6417 & 0.6307 \\ \hline
    auc & 0.7408 & 0.7312 \\ \hline
    rec & 0.7617 & 0.8375 \\ \hline
    \end{tabular}
\end{table}

\end{frame}

% -------------------- %
