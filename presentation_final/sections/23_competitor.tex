% -------------------- %

\begin{frame}{\lpn}
\framesubtitle{Competitor}

\begin{itemize}
    \item \lp{} for Non-stationary and Imbalanced Environments
    \item modified algorithm of \lpc
    \begin{itemize}
        \item employs a different penalty constraint that forces the algorithm to balance predictive accuracy on all classes
        \item uses a bagging based sub-ensemble for the minority class oversampling
    \end{itemize}
\end{itemize}

\begin{block}{For more details}
\small{[Gregory Ditzler and Robi Polikar] Incremental Learning of Concept Drift from Streaming Imbalanced Data. In IEEE Transactions on Knowledge and Data Engineering, vol. 25, no. 10, pages 2283 -- 2301. 10.1109/TKDE.2012.136, 2013.} 
\end{block}

\end{frame}

% -------------------- %

\begin{frame}{DWMIL vs. \lpn}
\framesubtitle{Forest Covertype}

\begin{itemize}
    \item DWMIL performs better at this real-world data set
\end{itemize}

\begin{table}[h]
    \centering
    \begin{tabular}{ | l | l | l | }
    \hline
    metric & DWMIL & \lpn \\ \hline \hline
    gm & 0.9222 & 0.8984 \\ \hline
    f1 & 0.8040 & 0.7838 \\ \hline
    auc & 0.9211 & 0.8928 \\ \hline
    rec & 0.9129 & 0.8616 \\ \hline
    \end{tabular}
\end{table}

\end{frame}

% -------------------- %

\begin{frame}{DWMIL vs. \lpn}
\framesubtitle{Moving Gaussian}

\begin{itemize}
    \item \lpn{} performs better at this data set
\end{itemize}

\begin{table}[h]
    \centering
    \begin{tabular}{ | l | l | l | }
    \hline
    metric & DWMIL & \lpn \\ \hline \hline
    gm & 0.7956 & 0.9520 \\ \hline
    f1 & 0.8354 & 0.9568 \\ \hline
    auc & 0.7964 & 0.9511 \\ \hline
    rec & 0.7823 & 0.9724 \\ \hline
    \end{tabular}
\end{table}

\end{frame}

% -------------------- %

\begin{frame}{DWMIL vs. \lpn}
\framesubtitle{SEA}

\begin{itemize}
    \item both methods perform equally good
\end{itemize}

\begin{table}[h]
    \centering
    \begin{tabular}{ | l | l | l | }
    \hline
    metric & DWMIL & \lpn \\ \hline \hline
    gm & 0.9388 & 0.9729 \\ \hline
    f1 & 0.9398 & 0.9734 \\ \hline
    auc & 0.9385 & 0.9727 \\ \hline
    rec & 0.9460 & 0.9796 \\ \hline
    \end{tabular}
\end{table}

\end{frame}

% -------------------- %

\begin{frame}{DWMIL vs. \lpn}
\framesubtitle{Hyper Plane}

\begin{itemize}
    \item \lpn{} performs better at this data set
\end{itemize}

\begin{table}[h]
    \centering
    \begin{tabular}{ | l | l | l | }
    \hline
    metric & DWMIL & \lpn \\ \hline \hline
    gm & 0.5751 & 0.9578 \\ \hline
    f1 & 0.5822 & 0.9594 \\ \hline
    auc & 0.5747 & 0.9575 \\ \hline
    rec & 0.5929 & 0.9662 \\ \hline
    \end{tabular}
\end{table}

\end{frame}

% -------------------- %

\begin{frame}{DWMIL vs. \lpn}
\framesubtitle{Checkerboard}

\begin{itemize}
    \item \lpn{} performs better at this data set
\end{itemize}

\begin{table}[h]
    \centering
    \begin{tabular}{ | l | l | l | }
    \hline
    metric & DWMIL & \lpn \\ \hline \hline
    gm & 0.6500 & 0.9484 \\ \hline
    f1 & 0.6534 & 0.9496 \\ \hline
    auc & 0.6497 & 0.9484 \\ \hline
    rec & 0.6574 & 0.9490 \\ \hline
    \end{tabular}
\end{table}

\end{frame}

% -------------------- %

\begin{frame}{DWMIL vs. \lpn}
\framesubtitle{Electricity}

\begin{itemize}
    \item \lpn{} performs better at this data set
\end{itemize}

\begin{table}[h]
    \centering
    \begin{tabular}{ | l | l | l | }
    \hline
    metric & DWMIL & \lpn \\ \hline \hline
    gm & 0.8026 & 0.9119 \\ \hline
    f1 & 0.7825 & 0.9064 \\ \hline
    auc & 0.7964 & 0.9068 \\ \hline
    rec & 0.7129 & 0.8607 \\ \hline
    \end{tabular}
\end{table}

\end{frame}

% -------------------- %

\begin{frame}{DWMIL vs. \lpn}
\framesubtitle{Weather}

\begin{itemize}
    \item both methods perform equally good
\end{itemize}

\begin{table}[h]
    \centering
    \begin{tabular}{ | l | l | l | }
    \hline
    metric & DWMIL & \lpn \\ \hline \hline
    gm & 0.7150 & 0.7731 \\ \hline
    f1 & 0.6447 & 0.7400 \\ \hline
    auc & 0.7162 & 0.7662 \\ \hline
    rec & 0.7234 & 0.6126 \\ \hline
    \end{tabular}
\end{table}

\end{frame}

% -------------------- %
