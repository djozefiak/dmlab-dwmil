% -------------------- %

\begin{frame}{Conclusion}
\framesubtitle{Authors of the paper}

\begin{itemize}
    \item concept drift and class imbalance are inevitable problems of learning from data streams
    \item DWMIL was proposed to solve these two problems
    \item the conducted experiments have shown that DWMIL
    \begin{itemize}
        \item performs better compared with its counterparts
        \item performs more efficiently compared with its counterparts
    \end{itemize}
\end{itemize}

\end{frame}

% -------------------- %

\begin{frame}{Conclusion}
\framesubtitle{Reproduction}

\begin{itemize}
    \item our conducted experiments have shown that DWMIL
    \begin{itemize}
        \item performs not always better compared with its counterparts
        \item performs not always more efficiently compared with its counterparts
    \end{itemize}
    \item \lpn{} performed better compared to DWMIL most of the times
    \item \lpn{} performed more efficiently compared to DWMIL most of the times
\end{itemize}

\end{frame}

% -------------------- %

\begin{frame}{Thanks for your attention!}
\framesubtitle{References}

\begin{itemize}
    \item{} [Yang Lu, Yiu-ming Cheung and Yuan Yan Tang] Dynamic Weighted Majority for Incremental Learning of Imbalanced Data Streams with Concept Drift. In Proceedings of the Twenty-Sixth International Joint Conference on Artificial Intelligence, pages 2393 -- 2399. IJCAI-17, 2017.
    \item{} [Gregory Ditzler and Robi Polikar] Incremental Learning of Concept Drift from Streaming Imbalanced Data. In IEEE Transactions on Knowledge and Data Engineering, vol. 25, no. 10, pages 2283 -- 2301. 10.1109/TKDE.2012.136, 2013.
\end{itemize}

\end{frame}

% -------------------- %